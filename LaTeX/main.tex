% Exercise Template
%A LaTeX template for typesetting exercise in Persian (with cover page).
%By: Zeinab Seifoori

\documentclass[12pt]{exam}

\usepackage{setspace}
\usepackage{listings}
\usepackage{graphicx,subfigure,wrapfig}
\usepackage{multirow}
%\usepackage{multicol}

\usepackage[margin=20mm]{geometry}
\usepackage{xepersian}
\usepackage{fontspec}
\settextfont[ExternalLocation]{XB Niloofar.ttf}
   
\newcommand{\class}{درس طراحی پایگاه داده‌ها}
\newcommand{\term}{نیم‌سال دوم ۰۱-۰۲}
\newcommand{\college}{دانشکده مهندسی کامپیوتر}
\newcommand{\prof}{استاد: دکتر اسدی}

\singlespacing 
\parindent 0ex

\lstset{
keywordstyle=\textbf,
identifierstyle=, 
stringstyle=\ttfamily,
commentstyle=\color{LimeGreen}, 
stringstyle=\ttfamily,
numberstyle=\footnotesize,
showstringspaces=false} 
\begin{document}


% -------------------------------------------------------
%  Thesis Information
% -------------------------------------------------------

\newcommand{\ThesisType}
{سمینار}  % پایان‌نامه / رساله
\newcommand{\ThesisDegree}
{کارشناسی ارشد گرایش معماری کامپیوتر}  % کارشناسی / کارشناسی ارشد / دکتری
\newcommand{\ThesisMajor}
{مهندسی کامپیوتر}  % مهندسی کامپیوتر
\newcommand{\ThesisTitle}
{پروژه درس طراحی پایگاه ‌داده‌ها}
\newcommand{\ThesisAuthor}
{ملیکا علیزاده - ثمین اکبری - معین آعلی}
\newcommand{\ThesisSupervisor}
{مهدی آخی}
\newcommand{\ThesisDate}
{نیم‌سال دوم 02-03}
\newcommand{\ThesisDepartment}
{دانشکده مهندسی کامپیوتر}
\newcommand{\ThesisUniversity}
{دانشگاه صنعتی شریف}

% -------------------------------------------------------
%  English Information
% -------------------------------------------------------

\newcommand{\EnglishThesisTitle}{Database-Design Course Project}


\pagestyle{empty}
\include{cover-page}

% These commands set up the running header on the top of the exam pages
\pagestyle{head}
\firstpageheader{}{}{}
\runningheader{صفحه \thepage\ از \numpages}{}{\class}
\runningheadrule
%\begin{tabular}{p{.7\textwidth} l}
%\multicolumn{2}{c}{\textbf{به نام خدا}}\\
%\multirow{2}{*}{\includegraphics[scale=0.2] {images/logo.png}} & \\ \\
%&  \textbf{\class}\\
%&  \textbf{\term}\\
%&  \textbf{\prof}\\ \\
% \textbf{\college} &  \\
%\end{tabular}\\

%\rule[1ex]{\textwidth}{.1pt}
%\textbf{تمرین سری پنجم}

%\rule[1ex]{\textwidth}{.1pt}
%\makebox[45mm]{\hrulefill}\\
\vspace{0pt}

• پرسش‌های خود را می‌توانید در  تالار ایجاد شده در سایت درس مطرح کنید.\\
•  نوشتن نام خود را فراموش نفرمایید.\\

\centering
\vspace{0pt}
\gradetablestretch{2}
\vqword{پرسش} \hqword{پرسش}
\vpword{بارم} \hpword{بارم}
\vsword{نمره} \hsword{نمره}
\vtword{جمع نمرات} \htword{جمع نمرات}
    \addpoints % required here by exam.cls, even though questions haven't started yet.  
{\small
    \gradetable[h]%[pages]  % Use [pages] to have grading table by page instead of question
}

%\baselineskip = 9mm

\begin{questions}
\pointpoints{نمره}{نمره}


\question[10]
کار برنامه زیر چیست؟ 
\begin{latin}
\begin{lstlisting}[language=C,breaklines=true]
int main()
{
    char s[80];
    int i;
    gets(s);
    for(i=0; s[i] != '\0'; i++)
    {
       if(s[i] == 'a') 
         continue;
       putchar(s[i]);
    }
}
\end{lstlisting}
\end{latin}


\question 
تابعی به نام 
\lr{max}
بنویسید که دو عدد صحیح را به عنوان آرگومان ورودی گرفته ماکزیمم آنها را برگرداند.
\begin{parts}
\part[10] با دو \lr{return}
\part[10] با یک \lr{return}
\end{parts}

\question[20]
تابعی بازگشتی بنویسید که عددی را به عنوان آرگومان گرفته (مثلاً $x$) سه به توان آنرا ($3^x$) محاسبه و برگرداند. اگر بازگشتی ننویسید، بخشی از نمره را از دست خواهید داد.

\question[20]
تابعی بنویسید که دو عدد صحیح را به عنوان آرگومان پذیرفته، حاصل ت
سیم اولی بر دومی را محاسبه و برگرداند. اگر عدد دوم صفر است پیام مناسبی چاپ کنید. اگر تابع ننویسید بخشی از نمره را از دست خواهید داد.

\question[30]
فرمول بسط سینوس به صورت زیر است:

\[ sin(x) = x - \frac{x^3}{3!} + \frac{x^5}{5!} - \frac{x^7}{7!} +  \cdots \]

برنامه‌ای برای محاسبه  مجموع ۲۰ جمله اول آن بنویسید.
در صورت تمایل می‌توانید تابعی برای محاسبه فاکتوریل بنویسید.

\end{questions}
\end{document}