\subsection*{\underline{جبر رابطه‌ای}}

در این بخش پاسخ پرسمان‌های زیر با استفاده از جبر رابطه‌ای آورده شده است:

\begin{enumerate}
	\item
	‌هواپیماهایی که در ۲۹ اسفند از تهران به مشهد می‌روند و بیشتر از ۵ صندلی خالی دارند را پیدا کنید.
	
	
	
	\setLTR
$
selectedAirplane =  \prod_{airplaneName}(\sigma_{COUNT(*)>5}(\gamma_{tripNumber}(\\ \sigma_{startDate = 1403.12.29 \ \land \ originCity = Tehran \ \land \ destinationCity = Mashhad}(\\ airTrip \bowtie_{tripNumber} transportReservation \bowtie_{accountID} cutomer)))) \\
$
\setRTL
	\rule{\linewidth}{0.05mm}
	
	\item
	هتل‌هایی که در تاریخ ۱ فروردین اتاق دو تخته‌ی خالی دارند و امکانات استخر و باشگاه و \linebreak امتیاز بالای ۴ دارند را پیدا کنید.
	
	
	\setLTR
$
selectedHotels =  \prod_{hotelID}( \\
\prod_{hotelID,RoomID}( \sigma_{(amenities \ LIKE \ 'pool \ gym \ \%') \land (score > 4) }(hotel \bowtie_{roomID} \sigma_{capacity=2}(room)))
\\ - \\
\prod_{hotelID,RoomID}(\\ \sigma_{(start\leq1403.01.01 \ \land \ start+duration \ge 1403.01.01 \ \land \ (state = Unpaid \lor Awaiting \lor Bought))}(hotelReservation \\ \bowtie_{hotelID} \sigma_{(amenities \ LIKE \ 'pool \ gym \ \%') \land (score > 4) }(hotel) \\ \bowtie_{roomID} \sigma_{capacity=2}(room))))
\\ \\$
\setRTL
	\rule{\linewidth}{0.05mm}
	
	
	\item
	میزان تخفیفی که مشتریان با استفاده از کد تخفیف  norouz  دریافت کرده‌اند را حساب کنید.
	
	\setLTR
$
\prod_{SUM(discountAmount)}(\\
\prod_{discountAmount}(\sigma_{code=norouz}(\\tripReservation\bowtie_{reserveID}walletWalletTransaction\bowtie_{discountCodeID}constantDiscountCode)) 
\\ \bigcup \\ 
\prod_{discountAmount}(\sigma_{code=norouz}(\\tripReservation\bowtie_{reserveID}walletWalletTransaction\bowtie_{discountCodeID}percentDiscountCode)) 
)
\\ \\$
\setRTL
	\rule{\linewidth}{0.05mm}
	
	
	\item
	تماس‌های پشتیبانی که در مورد هتل الماس بوده‌اند را پیداکنید.
	
	\setLTR
$
\prod_{phoneCallID}(phoneCall \bowtie_{ticketID}ticket\bowtie_{reserveID}hotelReservation \bowtie_{hotelID}\sigma_{name='Almas'}(hotel))
\\$
\setRTL
	\rule{\linewidth}{0.05mm}	 
	
	
	
	
	\item
	مجموع هزینه‌هایی که‌ به واسطه باشگاه مشتریان در ماه فروردین کسر شده‌است را بیابید.
	
	\setLTR
$
\prod_{SUM(discountAmount)}(\\
\prod_{discountAmount}(\\
\sigma_{(type \ LIKE \ 'customerclub\%') \land (1403.01.01 \le reservationDate \le 1403.01.31)}(\\tripReservation\bowtie_{reserveID}walletWalletTransaction\bowtie_{discountCodeID}constantDiscountCode)
)
\\ \bigcup \\
\prod_{discountAmount}(\\
\sigma_{(type \ LIKE \ 'customerclub\%') \land (1403.01.01 \le reservationDate \le 1403.01.31)}(\\tripReservation\bowtie_{reserveID}walletWalletTransaction\bowtie_{discountCodeID}percentDiscountCode)
)
)
\\ \\$
\setRTL
	\rule{\linewidth}{0.05mm}
	
	
	\item
	تعداد رزروهایی که در مدت معین پرداخت نشده، و لغو شده‌اند را بیابید.
	
	\setLTR
$
\prod_{COUNT(*)}( \\
	(\prod_{reserveID}(\sigma_{state=cancelled \ not \ paid }(hotelReservation))) 
	\\ \bigcup \\
	(\prod_{reserveID}(\sigma_{state=cancelled \ not \ paid }(tripReservation))) 
	)
\\$
\setRTL
	\rule{\linewidth}{0.05mm}
	
	
	
	
	\item
	تعداد مسافرین به تفکیک نوع سفر(قطار، اتوبوس، هواپیما) در تعطیلات نوروز(اول تا 13 فروردین ماه) را بیابید.
	
		\setLTR
$
A = \rho_{busTrip_customers/COUNT(*)}(\prod_{COUNT(*),state}(\\customer\bowtie_{customerID}(\\\sigma_{(state=paid\land 
1403.01.01 \leq	startDate \leq 1403.01.13)}(busTrip\bowtie_{tripNumber}tripReservation)))) \\ \\
B = \rho_{trainTrip_customers/COUNT(*)}(\prod_{COUNT(*),state}(\\customer\bowtie_{customerID}(\\\sigma_{(state=paid\land 
	1403.01.01 \leq	startDate \leq 1403.01.13)}(trainTrip\bowtie_{tripNumber}tripReservation))))\\ \\
C = \rho_{airTrip_customers/COUNT(*)}(\prod_{COUNT(*),state}(\\customer\bowtie_{customerID}(\\\sigma_{(state=paid\land 
	1403.01.01 \leq	startDate \leq 1403.01.13)}(airTrip\bowtie_{tripNumber}tripReservation)))) \\ \\
\prod_{airTrip_customers,trainTrip_customers,busTrip_customers}(A\bowtie_{state}B\bowtie_{state}C)
\\ \\$
\setRTL
	\rule{\linewidth}{0.05mm}
	
	
	
	
	\item
	آمار تعداد کنسلی رزروهای هتل‌ها در 5 شهر با بیشترین خرید بلیط به مقصد آنجا به‌ تفکیک ستاره هتل‌ها را بیابید.
	
	\setLTR
$
cityCount = \prod_{COUNT(*),destinationCity}(\gamma_{destinationCity}(\sigma_{state=paid}(tripReservation\bowtie_{tripID}trip))) \\ \\ 
sortedCityCount = \tau_{COUNT(*)}(cityCount) \\ \\
cityCancelledCount = \prod_{COUNT(*),destinationCity,hotel.type}(\gamma_{destinationCity}( \sigma_{state='cancelled\%'}(\\hotelReservation\bowtie_{hotelID}hotel))\bowtie_{hotel.city=sortedCityCount.destinationCity}sortedCityCount) \\ \\
\prod_{COUNT(*)}(\gamma_{hotel.type}(cityCancelledCount))
\\ \\$
\setRTL
	\rule{\linewidth}{0.05mm}	 
	
	
	
	\item
	همه مسافرانی که در پرواز $W1296$ در تاریخ 6 فروردین برای همان روز در هتلی با بیشترین اتاق خالی رزرو دارند.
	
	\setLTR
$
destinationHotels = \sigma_{airTripName='W1296'\land startDate=1403.01.06}(
hotel \bowtie_{city=destination_city}airTrip) \\ \\
emptyRooms = \prod_{roomID,hotelID}(
destinationHotel \bowtie_{hotelID}room)
-
\prod_{roomID,hotelID}(room\bowtie_{hotelID}destinationHotel\bowtie_{hotelID}hotelReservation) \\ \\
allHotelsEmptyRooms = \prod_{COUNT(*),hotelID}(\gamma_{hotelID}(emptyRooms)) \\ \\
maxEmpty = allHotelsEmptyRooms - \sigma_{
COUNT(*)\le COUNT2}(\\allHotelsEmptyRooms \times \rho_{
allHotelsEmptyRooms2/allHotelsEmptyRooms , COUNT2/COUNT(*)}(allHotelsEmptyRooms)) \\ \\
\prod_{userID}(\sigma_{airTripName='W1296'\land startDate=1403.01.06}(\\customer\bowtie_{userID=customerID}airTrip\bowtie_{hotelID}hotelReservation\bowtie_{hotelID}maxEmpty))
\\ \\$
\setRTL

	\rule{\linewidth}{0.05mm}
	
	
	
	\item
	مشتریانی را بیابید که برای تاریخ ۱ فروردین، بلیط به مقصد شهر بابلسر رزرو کرده‌اند و همچنین \linebreak با کد تخفیف ۳۰ هزار تومانی (که در بخش CRM به‌ آن اشاره شد) رزرو هتل خود را هم از سایت انجام داده‌اند.
	
	\setLTR
$
\prod_{accountID} (customer\bowtie_{customerID} (\\\sigma_{startDate=1403.01.01 \land destination=Babolsar \land state = paid} (trip \bowtie_{tripNumber}tripReservation)))
\\ \cap \\ 
\prod_{accountID}(\sigma_{amount=30\land type=convincing}(constantDiscount\bowtie_{customerID}\\(customer\bowtie_{customerID}(\sigma_{startDate=1403.01.01\land state=paid}(hotelReservation)))))
\\ \\$
\setRTL
	\rule{\linewidth}{0.05mm}	 
	
	
	
\end{enumerate}